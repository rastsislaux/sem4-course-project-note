\sectioncentered*{Введение}
\addcontentsline{toc}{section}{Введение}

Ранобе - это японский жанр литературы, который получил широкое распространение во всем мире благодаря своей уникальной концепции. Ранобе представляет собой легкую литературу с графическими изображениями, которые рассчитаны на молодую аудиторию. Она часто описывает истории героев, живущих в фантастических мирах, а также включает элементы приключений, фэнтези, романтики и мистики.

Включение ранобе в базу знаний по литературе важно, потому что ранобе являются важной частью японской культуры и литературы, популярны во всем мире и содержат уникальные элементы, которые могут быть полезными для исследования и анализа. Включение их в базу знаний поможет исследователям, ученым и переводчикам в изучении японской культуры и литературы, а также позволит более широкой аудитории узнать об этой форме литературы и оценить ее ценность.

Исходя из вышесказанного, очевидно, что разработка фрагмента по ранобе для базы знаний интеллектуальной системы по литературе -- актуальная и важная задача, таким образом поставлена цель данной курсовой работы: разработать фрагмент базы знаний по библиографии -- предметную область ранобе.

Для достижения данной цели поставлены следующие задачи:
\begin{itemize}
    \item проанализировать подходы к проектированию базы знаний интеллектуальной справочной системы по библиографии
    \item спроектировать базу знаний интеллектуальной справочной системы по библиографии
    \item разработать базу знаний интеллектуальной справочной системы по библиографии
    \item протестировать базу знаний интеллектуальной справочной системы по библиографии
\end{itemize}
