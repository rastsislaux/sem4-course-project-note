\sectioncentered*{Заключение}
\addcontentsline{toc}{section}{Заключение}

В данной курсовой работе был разработан фрагмент базы знаний интеллектуальной справочной системы по библиографии, сфокусированный на предметной области ранобе - японских легких романов. Работа была нацелена на создание базы знаний, которая предоставляла бы полезные и актуальные сведения о ранобе, их авторах, персонажах, жанрах и других характеристиках.

В процессе выполнения работы были выполнены следующие шаги:
\begin{itemize}
    \item определена иерархии предметных областей в базе знаний, чтобы обеспечить структурированное и связанное представление информации.
    \item формализованы знания о ранобе, включая авторов, иллюстраторов, год выпуска, жанры и обложки, что позволило создать подробные описания конкретных произведений.
    \item применена технологии OSTIS и язык внешнего представления SCs для построения фрагмента базы знаний. Это обеспечило гибкость и удобство разработки, а также совместную работу над проектом с использованием Git.
\end{itemize}

Результаты работы включают более 170 формализованных сущностей, включающих 15 ранобе, 30 авторов и иллюстраторов, 115 персонажей, 5 жанров, 2 концепции и другие элементы. База знаний предоставляет пользователю возможность получить детальную информацию о ранобе и связанных с ними аспектах.

Ожидаемый эффект от проделанной работы заключается в создании удобного и полезного инструмента для любителей ранобе, студентов, исследователей и других пользователей, которые хотят более подробно изучить и исследовать японскую литературу.

Разработанная система имеет перспективы практического использования в библиотеках, магазинах, исследовательских платформах, где пользователи могут искать и получать информацию о ранобе. Это значительно облегчит процесс поиска и выбора ранобе, а также позволит пользователям получать более полное представление о произведениях и их создателях.

Дальнейшее развитие разработанной системы может включать расширение базы знаний путем добавления новых ранобе, авторов и других элементов.

В целом, разработанный фрагмент базы знаний японских легких романов представляет собой важный шаг в создании интеллектуальной справочной системы, которая может быть полезной и информативной для широкого круга пользователей. Представленные результаты и методы работы могут быть использованы в дальнейших исследованиях и разработках в области литературных баз знаний, что дает возможность продолжить и углубить исследования в этой области.
