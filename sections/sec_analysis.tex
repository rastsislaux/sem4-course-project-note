\section{Анализ подходов к проектированию интеллектуальной справочной системы по библиографии}
\label{sec:analysis}

\subsection{Описание системы}

\subsubsection{Предметная область библиографии}

Предметная область библиографии - это та область знаний, которая изучает литературу как искусство и явление культуры. Она включает в себя изучение литературных жанров, тем и мотивов, стилей и техник, а также исторических, социальных и культурных контекстов, в которых литература возникает и развивается. Предметная область библиографии также включает в себя анализ литературных текстов, их содержания, стиля и формы, исследование творческого процесса и литературного творчества, а также критический анализ и оценку литературных произведений. Кроме того, предметная область библиографии включает в себя изучение взаимодействия литературы с другими искусствами и областями культуры, такими как кино, театр, музыка, философия, история и социология.

В целом, предметная область библиографии исследует литературу как явление культуры, которое имеет важное значение для понимания человеческой жизни, общества и культуры в целом.

\subsubsection{Предметная область ранобе}

Ранобе, или японские лёгкие новеллы, стали популярными в последние годы, как форма литературы, которая обычно написана в стиле повествования и часто включает элементы фэнтези, научной фантастики, романтики и других жанров. Включение ранобе в базу знаний по литературе важно по нескольким причинам:

\begin{itemize}
    \item Ранобе являются важной частью японской культуры и литературы. Они отражают японскую жизнь, историю и культуру, и позволяют читателям погрузиться в мир японской литературы.
    \item Ранобе популярны во всем мире, и многие люди находят в них удовольствие. Включение ранобе в базу знаний по литературе позволит более широкой аудитории узнать об этой форме литературы и оценить ее ценность.
    \item Ранобе содержат уникальные элементы, которые могут быть интересными и полезными для исследования и анализа. Это может включать в себя различные темы, стили и жанры, а также их влияние на другие формы литературы и культуру в целом.
    \item Включение ранобе в базу знаний по литературе может помочь исследователям и ученым в изучении японской литературы и культуры, а также помочь переводчикам в переводе их на другие языки.
\end{itemize}

Таким образом, включение ранобе в базу знаний по литературе может быть полезным для широкой аудитории, исследователей и переводчиков, и позволит более глубоко изучить японскую культуру и литературу. Поэтому, было решено разрабоать фрагмент базы знаний на тему <<Предметная область ранобе>> и были выделены следующие задачи:
\begin{itemize}
    \item Спроектировать фрагмент базы знаний для интеллектуальной справочной системы по библиографии, содержащий знания о предметной области ранобе.
    \item Реализовать фрагмент базы знаний для интеллектуальной справочной системы по библиографии, содержащий знания о предметной области ранобе.
    \item Протестировать фрагмент базы знаний для интеллектуальной справочной системы по библиографии, содержащий знания о предметной области ранобе.
\end{itemize}

\subsection{Подходы к разработке баз знаний}

Существует важное различие между терминами «данные» и «знания»: данные представляют собой отдельные факты, свойства объектов, процессов и явлений в определенной области, полученные в эмпирическом исследовании. Знания же, в свою очередь, основаны на этих данных, но создаются в результате мыслительной деятельности человека, который обобщает свой опыт, приобретенный в процессе практической работы. Они представляют собой закономерности этой предметной области (принципы, связи, законы), полученные благодаря практическому опыту и профессиональной деятельности специалистов, которые могут применять эти знания для решения конкретных задач. Можно также рассматривать знания как хорошо структурированные данные или как данные о данных (метаданные). \cite[с.~19-20]{KBIS_Gavrilova}

Существует множество различных моделей представления знаний, которые используются для разных предметных областей, однако большинство из них можно отнести к одним из четырех основных типов:
\begin{itemize}
    \item продукционные модели
    \item фреймовые модели
    \item формальные логические модели
    \item семантические модели \cite[с.~14-15]{Model_Baryshev}
\end{itemize}

\subsubsection{Продукционные модели}

Продукционная модель или модель, основанная на правилах, позволяет представить знания в виде предложений, типа <<если (\textit{условие}), то (\textit{действие})>>.

Под <<условием>> (\textit{антицедентом}) понимается некоторое предложение-образец, по которому осуществляется поиск в базе знаний, а под <<действием>> (\textit{консеквентом}) -- действия, выполняемые в успешном исходе поиска (они могут быть промежуточными, выступающими далее как условия и терминальными или целевыми, завершающими работу системы).

Чаще всего вывод на такой \textit{базе знаний} бывает \textbf{прямой} (от данных к поиску цели) или \textbf{обратный} (от цели её подтверждения -- к данным). \textit{Данные} в этом контексте -- исходные факты, хранящиеся в \textit{базе фактов}, на основании которых запускается \textit{машина вывода} или \textit{интерпретатор правил}, перебирающий правила из \textit{продукционной базы знаний}. \cite[с.~21-22]{KBIS_Gavrilova}

\textit{Системы продукций} с \textbf{прямыми выводами} являются одним из основных типов систем, использующих знания. Они включают три компонента: \textit{базу правил}, \textit{базу данных} и \textit{интерпретатор для логического вывода}. В процессе выполнения цикла «понимание – выполнение» база данных обновляется на основании выбранных правил, пока не будет достигнут целевой результат. Однако, с увеличением числа правил, скорость вывода замедляется, что ограничивает использование этих систем для крупномасштабных задач.

\textit{Система продукций} с \textbf{обратными выводами} использует правила для построения дерева, которое связывает факты и заключения. Оценка этого дерева основывается на фактах в базе данных, и результатом является логический вывод. Обратные выводы имеют преимущество в том, что они оценивают только те части дерева, которые имеют отношение к заключению. Однако, если отрицание или утверждение невозможны, то порождение дерева лишено смысла. В \textit{двунаправленных выводах} сначала оценивается небольшой объем полученных данных и выбирается гипотеза, а затем запрашиваются данные, необходимые для принятия решения о пригодности данной гипотезы. Это позволяет создать более мощную и гибкую систему. 

\textbf{Сильные стороны} продукционной модели:
\begin{itemize}
    \item простота создания и понимания отдельных правил
    \item простота модификаций и расширения базы знаний
    \item простота механизма логического вывода
    \item естественная модульность организации баз знаний
\end{itemize}

\textbf{Слабые стороны} продукционной модели:
\begin{itemize}
    \item сложность оценки целостного образа знаний
    \item крайне низкая эффективность обработки
    \item отличие от человеческой структуры знаний
    \item отсутствие гибкости в логическом выводе \cite[с.~15-16]{Model_Baryshev}
\end{itemize}

\subsubsection{Фреймовые модели}

\textit{Фрейм} -- это абстрактный образ для представленя некоего стереотипа восприятия. Термин \textit{фрейм} был предложен Марвином Минским, одним из пионеров искусственного интеллекта для обозначения структуры знаний для восприятия пространственных сцен. Эта модель, как и семантическая сеть, имеет глубокое психологическое обоснование.

Например, произнесение слова, такого как <<комната>>, вызывает у слушателей определенный образ в их воображении. Этот образ включает в себя основные характеристики комнаты, такие как четыре стены, пол, потолок, окна и дверь. Нельзя убрать эти основные характеристики, такие как окна, иначе мы перестаем говорить о комнате и начинаем говорить о другом объекте, например, чулане. Но этот абстрактный образ также имеет <<\textit{слоты}>> - неопределенные характеристики, такие как количество окон, цвет стен, высота потолка и т.д.

В теории фреймов такой образ комнаты называется фреймом комнаты. Фреймом называется также и формализованная модель для отображения образа. Различают \textit{фреймы-образцы} (\textit{прототипы}), хранящиеся в базе знаний и \textit{фреймы-экземпляры}, которые создаются для отображения реальных фактических ситуаций на основе поступающих данных. Модель фрейма является достаточно универсальной, поскольку позволяет отобразить всё многообразие знаний о мире через:
\begin{itemize}
    \item фреймы-\textit{структуры}, использующиеся для обозначения объектов и понятий (заём, залог...)
    \item фреймы-\textit{роли} (менеджер, кассир, клиент...)
    \item фреймы-\textit{сценарии} (банкротство, собрание, празднование...)
    \item фреймы-\textit{ситуации} (авария, тревога...)
\end{itemize}

Теория фреймов заимствует важное свойство из теории семантических сетей, которое называется "заимствование свойств". В обеих теориях наследование осуществляется через AKO-связи (A-Kind-Of = это), которые указывают на фрейм или понятие более высокого уровня в иерархии. Благодаря этому наследованию свойства передаются от более общего к более конкретному фрейму. \cite[с.~23-25]{KBIS_Gavrilova}

Оснонвым преимуществом фреймов как модели представления знаний является то, что она отражает концептуальную основу памяти человека, а также её гибкость и наглядность. \cite[с.~25]{KBIS_Gavrilova}. Также фреймы обеспечивают требования структурированности и связанности. Это достигается за счёт свойств наследования и вложенности, которыми обладают фреймы. \cite[с.~17]{Model_Baryshev}

\subsubsection{Формальные логические модели}

Логическая модель используется для представления знаний в системе логики предикатов первого порядка и логики высказываний. 

Формальная дедуктивная система, лежащая в основе логической модели представления знаний, может быть представлена в виде четверки f = <B, S, A, P>, где B -- множество базовых элементов (алфавит); S -- множество синтаксических правил, на основые которых из B строятся правильно построенные формулы; A -- множество аксиом (правильно построенных формул); P -- множество правил вывода (семантических правил), которые из множества аксиом позволят получать новые правильно построенные формулы -- теоремы.

Использование логики предикатов для представления знаний имеет ряд преимуществ. Во-первых, такой подход обладает четкими математическими свойствами и позволяет создавать мощные механизмы вывода, которые можно непосредственно запрограммировать. Во-вторых, такой подход обеспечивает контроль логической целостности базы знаний, что позволяет гарантировать ее непротиворечивость и полноту. В-третьих, запись фактов в такой модели очень проста и компактна.

Однако главным недостатком логической модели является отсутствие четких принципов организации фактов в базе знаний. Из-за этого обработка и анализ больших баз знаний может быть затруднительным. \cite[с.~15]{Model_Baryshev}

\subsubsection{Семантические модели}

Семантическая сеть - это орграф, в котором вершины представляют понятия, а дуги отображают отношения между ними. Термин \textit{семантическая} означает <<относящийся к смыслу>>, <<смысловая>>, а <<семантика>> -- это наука, которая определяет отношения между символами и объектами, которые они обозначают, то есть наука, определяющая смысл знаков. В качестве понятий в семантической сети обычно выступают абстрактные или конкретные объекты, а отношения -- это связи между этими объектами, типа: <<это>> (<<AKO -- A Kind Of>>, <<is>>), <<имеет частью>> (<<has part>>), <<принадлежит>>, <<любит>> и т.д. Характерной особенностью семантический сетей является обязательное наличие трёх типов отношений: <<\textbf{класс} -- элемент класса>>, <<\textbf{свойство} -- значение>>, <<\textbf{пример} элемента класса.>>.

Можно предположить несколько классификаций семантических сетей, связанных с типами отношений между понятиями: по количеству различных типов отношений (однородные -- с единственным типом отношений, неоднородные -- с различными типами отношений), по самим типам этих отношений (бинарные -- отношения связывают ровно два объекта, N-арные -- отношения могут свяывать два и более объектов).

Проблема поиска решения в базе знаний типа семантической сети сводится к задаче поиска фрагмента сети, соответствующего некоторой подсети, отражающей поставленный запрос к базе. \cite[с.~22-23]{KBIS_Gavrilova}

Достоинствами моделей представления знаний с помощью семантических сетей являются большие выразительные возможности, наглядность, близость структуры сети семантической структуре предметной области \cite[с.~18]{Model_Baryshev}, соответствие современным представлениям об организации долговременной памяти человека. \cite[с.~23]{KBIS_Gavrilova}

Семантические сети - это метод представления знаний, который имеет недостатки в теоретическом развитии логического вывода, что приводит к увеличению произвольности, введенной разработчиком. Поэтому процедуры вывода в таких системах могут быть противоречивыми, и необходимо уделять больше внимания анализу противоречивости. Эту функцию в семантических моделях зачастую выполняет человек, что усложняет решение задач при большом объеме знаний и ограничивает области применения. \cite[с.~18]{Model_Baryshev} Также недостатком семантической модели является сложность процедуры поиска вывода. \cite[с.~23]{KBIS_Gavrilova}

\subsection{Аналогичные проекты}

На данный момент уже существует определенное количество систем, предоставляющие пользователю услуги справочной системы по ранобе, такие как:
\begin{itemize}
    \item \textbf{Shikimori.one} -- русскоязычный сайт-база данных о японской анимации, манге и лёгкой прозе (ранобе). Он является крупнейшей русскоязычной платформой для обмена информацией о японской анимации и манге, а также для общения фанатов аниме и манги. \cite{Shikimori}
    \item \textbf{MyAnimeList} -- англоязычный сайт-база данных о японской анимации, манге и лёгкой прозе (ранобе). Он является одним из самых популярных ресурсов для обмена информацией о японской анимации и манге в мире. \cite{MAL}
\end{itemize}

Хотя MyAnimeList и Shikimori.one являются крупнейшими платформами для обмена информацией о японской анимации, манге и ранобе, они не могут выполнять функции интеллектуальной справочной системы по библиографии. Это связано с тем, что они не содержат информации о самой предметной области, такой как литература, история японской культуры и т.д. Они лишь предоставляют информацию о конкретных тайтлах, включая сюжет, персонажей и создателей, но не содержат аналитических материалов, комментариев и другой углубленной информации о предметной области.

\subsection{Выбор средств решения задачи}

В качестве средства решения поставленной задачи было решено выбрать семантическую модель базы знаний. В настоящее время идет активная работа над совершенствованием и проектированием технологий, способных решить задачу проектирования баз знаний на основе семантических сетей на должном уровне. Конечно, каждая технология обладает преимуществами и недостатками. Рассмотрим некоторые из таких технологий,
а также языки, используемые ими.

\subsubsection{Технологии построения семантических сетей}

\textbf{Semantic web} -- это технология, которая расширяет возможности существующего веба, объединяя данные из разных источников. Информация, которая доступна в Интернете, может быть прочитана человеком, но Semantic Web создан для того, чтобы сделать ее пригодной для автоматического анализа и синтеза выводов. Он использует способ представления данных в виде семантической сети с помощью онтологий, что отличается от классического веба. Семантическая паутина использует языки описания, такие как XML, XML Schema, RDF, RDF Schema, OWL и другие, для технической реализации. Она позволяет преобразовывать данные и выводы, основанные на этих данных, в различные представления, которые могут быть полезны на практике. \cite{SemanticWeb}

Преимущества Semantic Web:
\begin{itemize}
    \item высокая популярность и распространенность стандарта
    \item простота в изучении и использовании
\end{itemize}

Наиболее обсуждаемыми проблемами инструментария Semantic web являются:
\begin{itemize}
    \item неоднозначность при определении классов и их экземпляров;
    \item отсутствие возможности определять свойства у свойств, что не позволяет моделировать атрибуты у предметных отношений, n-арные отношения и атрибуты у атрибутов;
    \item ориентация на web и близкое к машинному представление семантических сетей;
    \item неразвитые стандарты представления переменных во времени и нечетких предметных областей;
    \item слабый уровень верификации онтологий на противоречивость и полноту. \cite{Kaeshko}
\end{itemize}

\textbf{OSTIS} -- это технология, основанная на семантических сетях, которая позволяет представлять знания. Она предназначена для компонентного проектирования систем, которые управляют знаниями. Предлагаемая технология оформляется как интеллектуальная метасистема, которая строится на основе предлагаемой технологии и содержит в себе все модели, средства и методы, связанные с данной технологией.

Информация представляется в виде логико-семантической модели, которая является платформенно-независимой и позволяет интеграцию с логико-семантическими моделями других систем. Она полностью отображает семантику используемых знаний и методов решения задач.

В процессе проектирования систем, управляемых знаниями, необходимо обращать особое внимание на технологию обновления систем в процессе их эксплуатации, а также на метатехнологию постоянного совершенствования самой технологии компонентного проектирования. Результатом проектирования логико-семантической модели любой интеллектуальной системы является текст, содержащий исходный текст базы знаний, исходные тексты программ и документацию. \cite{Model_Shunkevich}

Преимущества технологии OSTIS:
\begin{itemize}
    \item имеет строгую теоретико-множественную трактовку и не привязаны к конкретной прикладной области;
    \item имеет возможность представления отношений любой арности;
    \item отношения представляются в виде узлов семантической сети, что позволяет характеризовать их свойства;
    \item экземпляры отношений выделяются как отдельные узлы семантической сети, что дает возможность характеризовать каждый экземпляр отношения уникальным образом;
    \item в алфавите ключевых узлов и дуг имеются элементы для описания нечетких, негативных и нестационарных объектов. \cite{Kaeshko}
\end{itemize}

Недостатки технологии OSTIS --- это малое количество доступной документации, а также малое количество обучающих материалов с низким порогом вхождения, низкая популярность и распространенность стандарта в мире, высокая ресурсоемкость и низкая стабильность работы ostis-систем на данный момент.

\subsubsection{Языки построения семантических сетей}

\textbf{OWL} является языком для создания и определения веб-онтологий. Он описывает классы, свойства и их экземпляры, а также отношения между ними. OWL используется для явного представления смысла терминов и словарных отношений. Язык OWL предоставляет три подязыка, каждый из которых предназначен для конкретных сообществ исполнителей и пользователей:

\begin{itemize}
    \item OWL Lite -- наименее выразительный, подходит для простых иерархий классов и ограничений, удобен для миграции тезаурусов и таксономий.
    \item OWL DL -- более выразительный, сохраняет вычислительную полноту, гарантирует вычислимость выводов и основан на логическом описании.
    \item OWL Full -- наиболее выразительный, имеет синтаксическую свободу RDF, не обеспечивает вычислительные гарантии, но позволяет расширять смысл заранее определенных словарей. \cite{Kursk}
\end{itemize}

OWL является развитием технологий XML и RDF и входит в новый стек веб-протоколов. В отличие от многих языков представления знаний, OWL не является исключительно академическим, а представляет практическую ценность. Он получает финансовую поддержку и имеет множество программных инструментов для работы.

С точки зрения полноты выразительных возможностей язык OWL не может конкурировать со многими существующими языками представления знаний уже хотя бы потому, что не рассчитан на описание динамических явлений (его основа -- это объекты и связи). \cite{Kaftannikov}

\textbf{SC-код} -- используется для организации сложной структурированной информации, путем унификации различных видов знаний в общую форму. Этот принцип позволяет интегрировать знания в базе знаний и объединять их. Существуют различные внешние представления SC-кода, такие как SCg, который представляет содержимое базы знаний в виде графа; SCn, приближенный к естественному языку; SCs, который представляет содержимое базы знаний в виде строки.

Достоинствами SC-кода являются следующие его свойства:
\begin{itemize}
    \item Неограниченная возможность перехода от sc-текстов к sc-метатекстам, содержащим знаки описываемых sc-текстов;
    \item Тексты SC-кода могут быть иерархическими структурами, поскольку sc-элемент может обозначать множество, состоящее из любых sc-элементов;
    \item Все основные семантические связи между текстами в SC-коде становится теоретико-множественными.
\end{itemize}

SC-код объединяет в себе язык и метаязык и может использоваться для описания синтаксиса, семантики и онтологии этого кода. Он работает только с семантически нормализованными множествами, что позволяет четко разделять первичные (терминальные) элементы, которые являются обозначениями внешних объектов, и вторичные элементы, которые являются обозначениями множеств и могут быть использованы в составе SC-конструкций. После формирования синтаксиса строится модель, которая может описывать фрагмент какой-либо предметной области. \cite[с~116-117]{OSTIS}

\subsection{Вывод}
В результате анализа можно сделать следующие выводы:

\begin{itemize}
    \item Предметная область ранобе на данный момент не представлена в виде публично доступой базы знаний, несмотря на популярность ранобе и её важность для японской и мировой культур и литератур -- потому решено построить фрагмент базы знаний, содержащей знания о предметной области ранобе в рамках интеллектуальной справочной системы по библиографии.
    \item В связи с преимуществами семантической модели представления знаний, для построения фрагмента базы знаний было решено использовать технологию OSTIS.
    \item Из языков внешнего представления SC-кода был выбран SCs (Semantic Code string), поскольку он наиболее удобен для совместной разработки с использованием Git.
\end{itemize}
